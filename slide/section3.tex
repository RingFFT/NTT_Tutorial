\section{Decomposition of \( \mathbb{Z}_{p}[x]/(x^{n}-1) \)}
\subsection{Decomposition of \( \mathbb{Z}_{p}[x]/(x^{8}-1) \)}
\begin{frame}
    \begin{itemize}
        \item <1->Proceeding from the previous example, we now try to decompose the ring 
            \( \mathbb{Z}_{p}[x]/(x^{8}-1) \).
        \item <2->We assume that the modulus number \( p \) will make the existence of 
            8-th primitive root \( \omega_{8} \), that is, \( \omega_{8}^{8} = 1 \) and
            \( \omega_{8}^{4} =  -1 \). 
            Existence of \( \omega_{4} \) follows from \( \omega_{4} = \omega_{8}^{2} \).
        \item <3->We have
            \begin{align*}
                (x^{8}-1) &= (x^{4}-1)(x^{4}+1) = (x^{4} -1)(x^{4} - \omega_{4}^{2})\\
                          &= (x^{2}-1)(x^{2}+1)(x^{2} - \omega_{4})(x^{2} + \omega_{4})
                           = (x^{2}-1)  (x^{2}-\omega_{4}^{2})  (x^{2} - \omega_{4})(x^{2} + \omega_{4})\\
                          &= (x^{2}-1)(x^{2} - \omega_{4}^{2})(x^{2} - \omega_{8}^{2})(x^{2} + \omega_{8}^{2})
                          = (x^{2}-1)(x^{2} - \omega_{4}^{2})(x^{2} - \omega_{8}^{2})(x^{2} {\color{red}- \omega_{8}^{6}})\\
                          &= (x - 1)(x + 1)
                             (x - \omega_{4})(x + \omega_{4})
                             (x - \omega_{8})(x + \omega_{8})
                             (x - \omega_{8}^{3})(x + \omega_{8}^{3}).
            \end{align*}
        \item <4->Hence, we have the decomposition:
\begin{align*}
\mathbb{Z}_{p}[x]/(x^{8}-1)
&\cong (\mathbb{Z}_{p}[x]/(x^{4}-1))
        \times
        (\mathbb{Z}_{p}[x]/(x^{4}+1)) \\[6pt]
&\cong (\mathbb{Z}_{p}[x]/(x^{2}-1))
        \times
        (\mathbb{Z}_{p}[x]/(x^{2}+1))
        \times
        (\mathbb{Z}_{p}[x]/(x^{2}-\omega_{4}))
        \times
        (\mathbb{Z}_{p}[x]/(x^{2}+\omega_{4})) \\[6pt]
&\cong (\mathbb{Z}_{p}[x]/(x-1))
        \times
        (\mathbb{Z}_{p}[x]/(x+1))
        \times
        (\mathbb{Z}_{p}[x]/(x-\omega_{4}))
        \times
        (\mathbb{Z}_{p}[x]/(x+\omega_{4})) \\[3pt]
&\quad\times (\mathbb{Z}_{p}[x]/(x-\omega_{8}))
        \times
        (\mathbb{Z}_{p}[x]/(x+\omega_{8}))
        \times
        (\mathbb{Z}_{p}[x]/(x-\omega_{8}^{3}))
        \times
        (\mathbb{Z}_{p}[x]/(x+\omega_{8}^{3})).
\end{align*}
    \end{itemize}
\end{frame}

\begin{frame}
    \begin{itemize}
        \item The decomposition can be further written as:
\               \begin{align*}
    \only<1->{\mathbb{Z}_{p}[x]/(x^{8}-1)}
    &\only<2->{\cong(\mathbb{Z}_{p}[x]/(x^{4}-1))
            \times
            (\mathbb{Z}_{p}[x]/(x^{4}+1)) }
     \only<6->{{\color{red}= (\mathbb{Z}_{p}[x]/(x-\omega_{2}^{0}))
            \times
            (\mathbb{Z}_{p}[x]/(x-\omega_{2}^{1}))}} \\[6pt]
    &\only<3->{\cong(\mathbb{Z}_{p}[x]/(x^{2}-1))
            \times
            (\mathbb{Z}_{p}[x]/(x^{2}+1))
            \times
            (\mathbb{Z}_{p}[x]/(x^{2}-\omega_{4}))
            \times
            (\mathbb{Z}_{p}[x]/(x^{2}+\omega_{4}))} \\[6pt]
    &\only<7->{\color{blue}= (\mathbb{Z}_{p}[x]/(x^{2}-\omega_{4}^{0}))
            \times
            (\mathbb{Z}_{p}[x]/(x^{2}-\omega_{4}^{2}))
            \times
            (\mathbb{Z}_{p}[x]/(x^{2}-\omega_{4}))
            \times
            (\mathbb{Z}_{p}[x]/(x^{2}-\omega_{4}^{3}))} \\[6pt]
    &\only<4->{\cong(\mathbb{Z}_{p}[x]/(x-1))
            \times
            (\mathbb{Z}_{p}[x]/(x+1))
            \times
            (\mathbb{Z}_{p}[x]/(x-\omega_{4}))
            \times
            (\mathbb{Z}_{p}[x]/(x+\omega_{4}))} \\[3pt]
    &\quad\only<5->{\times (\mathbb{Z}_{p}[x]/(x-\omega_{8}))
            \times
            (\mathbb{Z}_{p}[x]/(x+\omega_{8}))
            \times
            (\mathbb{Z}_{p}[x]/(x-\omega_{8}^{3}))
            \times
            (\mathbb{Z}_{p}[x]/(x+\omega_{8}^{3}))} \\[6pt]
    &\quad\only<8->{\color{cyan}= (\mathbb{Z}_{p}[x]/(x-\omega_{8}^{0}))
            \times
            (\mathbb{Z}_{p}[x]/(x-\omega_{8}^{4}))
            \times
            (\mathbb{Z}_{p}[x]/(x-\omega_{8}^{2}))
            \times
            (\mathbb{Z}_{p}[x]/(x-\omega_{8}^{6}))} \\[3pt]
    &\quad\only<9->{\color{cyan}\times (\mathbb{Z}_{p}[x]/(x-\omega_{8}))
            \times
            (\mathbb{Z}_{p}[x]/(x-\omega_{8}^{5}))
            \times
            (\mathbb{Z}_{p}[x]/(x-\omega_{8}^{3}))
            \times
            (\mathbb{Z}_{p}[x]/(x-\omega_{8}^{7}))}.
    \end{align*}

        \item <10->It can then be sofisticatedly written as:
            \[
                \mathbb{Z}_{p}[x]/(x^{8}-1) 
                \only<11->{\cong \prod_{k=0}^{1} \mathbb{Z}_{p}[x]/(x-\omega_{2}^{\mathrm{brv}_{1}(k)})}
                \only<12->{\cong \prod_{k=0}^{3} \mathbb{Z}_{p}[x]/(x-\omega_{4}^{\mathrm{brv}_{2}(k)})}
                \only<13->{\cong \prod_{k=0}^{7} \mathbb{Z}_{p}[x]/(x-\omega_{8}^{\mathrm{brv}_{3}(k)}).}
            \]
            The notation \( \mathrm{brv}_{1}(k) \), \( \mathrm{brv}_{2}(k) \), and \( \mathrm{brv}_{3}(k) \) will be discussed in the next slide.

    \end{itemize}
\end{frame}

\begin{frame}
    \begin{itemize}
        \item <1->The notation \( \mathrm{brv}_{1}(k) \), \( \mathrm{brv}_{2}(k) \), and \( \mathrm{brv}_{3}(k) \) are referred to as bit-reversal permutation.
        \item <2->\( \mathrm{brv}_{j}(k) \) is defined as:
            \[
                \mathrm{brv}_{j}(k) = \sum_{i=0}^{j-1} b_{i}2^{j-1-i},
            \]
            where \( b_{i} \) is the \( i \)-th bit of the binary representation of \( k \).

        \item <3->If you think that the above formula is too complicated, it is equivalent to:
        \begin{enumerate}
            \item Write the number \( k \) in binary representation.
            \item Pad the binary representation with zeros to make it \( j \)-bit long.
            \item Reverse the bitstring.
            \item Convert the reversed bitstring to decimal.
        \end{enumerate}

        \item <4->You can check this
        \begin{align*}
        \prod_{k=0}^{7} \mathbb{Z}_{p}[x]/(x-\omega_{8}^{\mathrm{brv}_{3}(k)})
        \end{align*}
        equals to
        \begin{align*}
        &(\mathbb{Z}_{p}[x]/(x-\omega_{8}^{0}))
        \times
        (\mathbb{Z}_{p}[x]/(x-\omega_{8}^{4}))
        \times
        (\mathbb{Z}_{p}[x]/(x-\omega_{8}^{2}))
        \times
        (\mathbb{Z}_{p}[x]/(x-\omega_{8}^{6})) \\[3pt]
  &\quad\times (\mathbb{Z}_{p}[x]/(x-\omega_{8}))
        \times
        (\mathbb{Z}_{p}[x]/(x-\omega_{8}^{5}))
        \times
        (\mathbb{Z}_{p}[x]/(x-\omega_{8}^{3}))
        \times
        (\mathbb{Z}_{p}[x]/(x-\omega_{8}^{7}))
        \end{align*}

    \end{itemize}
\end{frame}

\begin{frame}
    \begin{itemize}
        \item <1->Now we can finally state the decomposition formula of \( \mathbb{Z}_{p}[x]/(x^{n}-1) \):
            \begin{align*}
                \mathbb{Z}_{p}[x]/(x^{n}-1) 
                & \cong \prod_{k=0}^{1} \mathbb{Z}_{p}[x]/(x^{\frac{n}{2}}-\omega_{2}^{\mathrm{brv}_{1}(k)})
                 \cong \prod_{k=0}^{3} \mathbb{Z}_{p}[x]/(x^{\frac{n}{4}}-\omega_{4}^{\mathrm{brv}_{2}(k)})\\
                & \cong \prod_{k=0}^{7} \mathbb{Z}_{p}[x]/(x^{\frac{n}{8}}-\omega_{8}^{\mathrm{brv}_{3}(k)})\\
                & \quad \vdots\\
                & \cong \prod_{k=0}^{n-1} \mathbb{Z}_{p}[x]/(x-\omega_{n}^{\mathrm{brv}_{\log_{2}n}(k)}).
            \end{align*}
            We will say that this is a \( \log_{2}n \)-level decomposition.
        \item <2->An important observation is that:
        In order to make the decomposition until the \( \log_{2}n \)-level, we need to have the existence of \( n \)-th primitive root.
        If, say, the current coefficient ring only has \( 4 \)-th primitive root, then we can only decompose until the \( 2 \)-level:
        \[
            \mathbb{Z}_{p}[x]/(x^{n}-1) 
            \cong \prod_{k=0}^{1} \mathbb{Z}_{p}[x]/(x^{\frac{n}{2}}-\omega_{2}^{\mathrm{brv}_{1}(k)}).
        \]
    \end{itemize}
\end{frame}

\begin{frame}
    \begin{itemize}
        \item An important observation is that:
        In order to make the decomposition until the \( \log_{2}n \)-level, we need to have the existence of \( n \)-th primitive root.
        If, say, the current coefficient ring only has \( 4 \)-th primitive root, then we can only decompose until the \( 2 \)-level:
        \[
            \mathbb{Z}_{p}[x]/(x^{n}-1) 
            \cong \prod_{k=0}^{1} \mathbb{Z}_{p}[x]/(x^{\frac{n}{2}}-\omega_{2}^{\mathrm{brv}_{1}(k)}).
        \]
        This is the so-called incomplete NTT. 
        Though not fully decoposed, it is still beneficial (sometimes better) to our purpose.
        Again, the performance should be measured by the cycle-count.
    \end{itemize}
\end{frame}






\begin{frame}
asd    
\end{frame}
