\section{Decomposition of \( \mathbb{Z}_{p}[x]/(x^{n}-1) \)}
\subsection{Decomposition of \( \mathbb{Z}_{p}[x]/(x^{8}-1) \)}
\begin{frame}
    \begin{itemize}
        \item Proceeding from the previous example, we now try to decompose the ring 
            \( \mathbb{Z}_{p}[x]/(x^{8}-1) \).
        \item We assume that the modulus number \( p \) will make the existence of 
            8-th primitive root \( \omega_{8} \), that is, \( \omega_{8}^{8} = 1 \) and
            \( \omega_{8}^{4} =  -1 \). 
            Existence of \( \omega_{4} \) follows from \( \omega_{4} = \omega_{8}^{2} \).
        \item We have
            \begin{align*}
                (x^{8}-1) &= (x^{4}-1)(x^{4}+1) = (x^{4} -1)(x^{4} - \omega_{4}^{2})\\
                          &= (x^{2}-1)(x^{2}+1)(x^{2} - \omega_{4})(x^{2} + \omega_{4})
                           = (x^{2}-1)  (x^{2}-\omega_{4}^{2})  (x^{2} - \omega_{4})(x^{2} + \omega_{4})\\
                          &= (x^{2}-1)(x^{2} - \omega_{4}^{2})(x^{2} - \omega_{8}^{2})(x^{2} + \omega_{8}^{2})
                          = (x^{2}-1)(x^{2} - \omega_{4}^{2})(x^{2} - \omega_{8}^{2})(x^{2} {\color{red}- \omega_{8}^{6}})\\
                          &= (x - 1)(x + 1)
                             (x - \omega_{4})(x + \omega_{4})
                             (x - \omega_{8})(x + \omega_{8})
                             (x - \omega_{8}^{3})(x + \omega_{8}^{3}).
            \end{align*}
        \item Hence, we have the decomposition:
\begin{align*}
\mathbb{Z}_{p}[x]/(x^{8}-1)
&\cong (\mathbb{Z}_{p}[x]/(x^{4}-1))
        \times
        (\mathbb{Z}_{p}[x]/(x^{4}+1)) \\[6pt]
&\cong (\mathbb{Z}_{p}[x]/(x^{2}-1))
        \times
        (\mathbb{Z}_{p}[x]/(x^{2}+1))
        \times
        (\mathbb{Z}_{p}[x]/(x^{2}-\omega_{4}))
        \times
        (\mathbb{Z}_{p}[x]/(x^{2}+\omega_{4})) \\[6pt]
&\cong (\mathbb{Z}_{p}[x]/(x-1))
        \times
        (\mathbb{Z}_{p}[x]/(x+1))
        \times
        (\mathbb{Z}_{p}[x]/(x-\omega_{4}))
        \times
        (\mathbb{Z}_{p}[x]/(x+\omega_{4})) \\[3pt]
&\quad\times (\mathbb{Z}_{p}[x]/(x-\omega_{8}))
        \times
        (\mathbb{Z}_{p}[x]/(x+\omega_{8}))
        \times
        (\mathbb{Z}_{p}[x]/(x-\omega_{8}^{3}))
        \times
        (\mathbb{Z}_{p}[x]/(x+\omega_{8}^{3})).
\end{align*}
    \end{itemize}
\end{frame}

\begin{frame}
    \begin{itemize}
        \item The decomposition can be further written as:
\begin{align*}
\mathbb{Z}_{p}[x]/(x^{8}-1)
&\cong (\mathbb{Z}_{p}[x]/(x^{4}-1))
        \times
        (\mathbb{Z}_{p}[x]/(x^{4}+1)) 
        {\color{red}= (\mathbb{Z}_{p}[x]/(x-\omega_{2}^{0}))
        \times
        (\mathbb{Z}_{p}[x]/(x-\omega_{2}^{1}))}
        \\[6pt]
&\cong (\mathbb{Z}_{p}[x]/(x^{2}-1))
        \times
        (\mathbb{Z}_{p}[x]/(x^{2}+1))
        \times
        (\mathbb{Z}_{p}[x]/(x^{2}-\omega_{4}))
        \times
        (\mathbb{Z}_{p}[x]/(x^{2}+\omega_{4})) \\[6pt]
&{\color{blue}=      (\mathbb{Z}_{p}[x]/(x^{2}-\omega_{4}^{0}))
        \times
        (\mathbb{Z}_{p}[x]/(x^{2}-\omega_{4}^{2}))
        \times
        (\mathbb{Z}_{p}[x]/(x^{2}-\omega_{4}))
        \times
        (\mathbb{Z}_{p}[x]/(x^{2}-\omega_{4}^{3}))} \\[6pt]
&\cong (\mathbb{Z}_{p}[x]/(x-1))
        \times
        (\mathbb{Z}_{p}[x]/(x+1))
        \times
        (\mathbb{Z}_{p}[x]/(x-\omega_{4}))
        \times
        (\mathbb{Z}_{p}[x]/(x+\omega_{4})) \\[3pt]
&\quad\times (\mathbb{Z}_{p}[x]/(x-\omega_{8}))
        \times
        (\mathbb{Z}_{p}[x]/(x+\omega_{8}))
        \times
        (\mathbb{Z}_{p}[x]/(x-\omega_{8}^{3}))
        \times
        (\mathbb{Z}_{p}[x]/(x+\omega_{8}^{3})) \\[6pt]
&{\color{cyan}=      (\mathbb{Z}_{p}[x]/(x-\omega_{8}^{0}))
        \times
        (\mathbb{Z}_{p}[x]/(x-\omega_{8}^{4}))
        \times
        (\mathbb{Z}_{p}[x]/(x-\omega_{8}^{2}))
        \times
    (\mathbb{Z}_{p}[x]/(x-\omega_{8}^{6}))} \\[3pt]
&\quad{\color{cyan}\times (\mathbb{Z}_{p}[x]/(x-\omega_{8}))
        \times
        (\mathbb{Z}_{p}[x]/(x-\omega_{8}^{5}))
        \times
        (\mathbb{Z}_{p}[x]/(x-\omega_{8}^{3}))
        \times
        (\mathbb{Z}_{p}[x]/(x-\omega_{8}^{7}))}.
\end{align*}

        \item It can then be sofisticatedly written as:
            \[
                \mathbb{Z}_{p}[x]/(x^{8}-1) 
                \cong \prod_{k=0}^{7} \mathbb{Z}_{p}[x]/(x-\omega_{8}^{\mathrm{brv}_{3}(k)}).
            \]


    \end{itemize}
\end{frame}


\begin{frame}
    \begin{itemize}
        \item By using such decomposition, we can develop a fast algorithm for 
            polynomial multiplication in the ring \( \mathbb{Z}_{p}[x]/(x^{8}-1) \).
        \item But the projection (and recombination) are more complicated than the 
            previous example. Hence we introduce the concept of butterfly algorithm.
        \item It is natural to represent a polynomial by an array in the programming language.
            So here we denote the polynomial \( a(x) = a_{0} + a_{1}x + \cdots + a_{7}x^{7} \) by 
            an array
            \[
                \begin{bmatrix}
                    a_{0} & a_{1} & a_{2} & a_{3} & a_{4} & a_{5} & a_{6} & a_{7}
                \end{bmatrix}.
            \]

    \end{itemize}
\end{frame}

